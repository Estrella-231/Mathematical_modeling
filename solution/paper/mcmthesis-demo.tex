
\documentclass{mcmthesis}
 %\documentclass[CTeX = true]{mcmthesis}  % 当使用 CTeX 套装时请注释上一行使用该行的设置
\mcmsetup{tstyle=\color{red}\bfseries,%修改题号,队号的颜色和加粗显示,黑色可以修改为 black
        tcn = 2631703, problem = C, %修改队号,参赛题号
        sheet = true, titleinsheet = true, keywordsinsheet = true,%修改sheet显示信息
        titlepage = false, abstract = true}

  %四款字体可以选择
  \usepackage{times}
  \usepackage{ctex}
  %\usepackage{newtxtext,newtxmath} %CTeX 无此字体,可用 txfonts 替代,请使用新版 TeXLive.
  %\usepackage{palatino}
  %\usepackage{txfonts}

\usepackage{indentfirst}  %首行缩进,注释掉,首行就不再缩进。
\usepackage{lipsum}
\title{Estimation of Fan Voting and Optimization of Voting Mechanism}
\author{\small \href{https://www.latexstudio.net/}
  {\includegraphics[width=7cm]{mcmthesis-logo}}}
\date{\today}
\begin{document}
\begin{abstract}
\par Dancing with the Stars , the U.S. version of the international TV franchise based on Britain’s “Strictly Come Dancing,” has completed 34 seasons. Celebrities paired with professional dancers perform weekly, with elimination and final rankings determined by combining expert judges’ technical scores (1-10 points, with subjectivity) and fan votes (phone/online, multiple votes allowed, influenced by popularity and charisma).
Two core methods have been used to combine votes: rank-based combination (Seasons 1, 2, 28-34) and percentage-based combination (Seasons 3-27). Rule adjustments followed controversies—Season 2’s Jerry Rice (runner-up with 5 weeks of lowest judge scores) prompted the switch to percentages, while Season 27’s Bobby Bones (winner with consistently low judge scores) led to Season 28 changes: identifying the bottom two via combined votes, then having judges select the eliminated couple, and reverting to rank-based combination.
Provided data covers 34 seasons, including contestants’ characteristics (age, industry), weekly judge scores, and season results, with notes on special values 
Key tasks include: 1) Developing mathematical models to estimate unknown fan votes, with consistency (aligning with weekly eliminations) and certainty measures; 2) Comparing the two combination methods across seasons, analyzing their impact on controversial cases, and recommending optimal methods (including whether to retain the “bottom two judge selection” rule); 3) Modeling how professional dancers and celebrity characteristics influence competition outcomes and their differential impact on judge scores vs. fan votes; 4) Proposing a more fair/exciting alternative voting system.

\begin{keywords}
vote combination optimization;estimate
\end{keywords}
\end{abstract}
\maketitle
%% Generate the Table of Contents, if it's needed.
\tableofcontents
\newpage

\section{Introduction}
\subsection{Background}
《与星共舞》(DWTS)美国版已播 34 季,名人与专业舞者配对参赛,结合评委技术得分(1-10 分)与粉丝投票(可多次投)决定淘汰 / 排名。历史采用两种投票结合方式:排名法(季 1-2、28-34)与百分比法(季 3-27),规则调整源于 “低评委分高排名” 争议(如季 2 Jerry Rice、季 27 Bobby Bones),季 28 后新增 “评委从末两位选淘汰者” 规则。提供 34 季选手特征、每周评委得分等数据,核心任务为估计未知粉丝投票、对比两种结合方式、分析影响因素及设计新投票系统。

\subsection{Restatement of the problem}
\begin{itemize}
\item 粉丝投票估计建模,构建数学模型估计每位选手参赛每周的未知粉丝投票数;验证模型与每周淘汰结果的一致性(提供度量指标);量化估计结果的确定性及其是否因选手 / 周次存在差异(提供度量指标)。
\item投票结合方式分析与推荐:跨赛季对比 “排名法” 和 “百分比法” 的结果差异,判断哪种更偏向粉丝投票;将两种方法应用于争议案例,分析结果是否改变,以以及 “评委从末两位选淘汰者” 规则的影响;推荐未来适用的结合方式及是否保留上述附加则
\item 影响因素建模:分析专业舞者及名人特征(年龄、行业等)对比赛成绩的影响程度;对比这些因素对评委得分和粉丝投票的影响是否一致。
新投票系统设计:提出更公平或更能提升节目精彩度的每周投票结合系统;提供该系统应被采纳的支持理由.
\end{itemize}

\subsection{Our work}

\section{Evaluate of the Mode}

\section{Strengths and weaknesses}


\subsection{Strengths}

\subsection{How to cite?}


\begin{thebibliography}{99}
\bibitem{1} D.~E. KNUTH   The \TeX{}book  the American
Mathematical Society and Addison-Wesley
Publishing Company , 1984-1986.
\bibitem{2}Lamport, Leslie,  \LaTeX{}: `` A Document Preparation System '',
Addison-Wesley Publishing Company, 1986.
\bibitem{3}\url{https://www.latexstudio.net/}
\end{thebibliography}
%%%%%%%%%%%%%%%%%%%%%%%%%%%%%%%%%%%%%%%%%%%%%%%%%%%%%%%%%%%%%%%%%%%%%%%%%%%%%%%%%%%附录
\begin{appendices}

\section{First appendix}

\section{Second appendix}

\end{appendices}

%%%%%%%%%%%%%%%%%%%%%%%%%%%%%%%%%%%%%%%%%%%%%%%%%%%%%%%%%%AI报告
\AImatter
\begin{ReportAiUse}{9}
\bibitem{AI1}
Bing AI\\
Query1: write a sort algorithm\\
Output: Sure, I can help you with that. Here is a simple implementation of the selection sort algorithm in Python:
\begin{lstlisting}[language=python]
def selection_sort(arr):
    for i in range(len(arr)):
        min_idx = i
        for j in range(i + 1, len(arr)):
            if arr[min_idx] > arr[j]:
                min_idx = j
        arr[i], arr[min_idx] = arr[min_idx], arr[i]
    return arr
\end{lstlisting}
This algorithm sorts an array by repeatedly finding the minimum element from the unsorted part of the array and putting it at the beginning. \href{https://www.geeksforgeeks.org/sorting-algorithms/}{The time complexity of this algorithm is O($n^2$)}.

There are many other sorting algorithms available such as bubble sort, insertion sort, merge sort, quick sort, heap sort, and more. Each of these algorithms has its own advantages and disadvantages. You can learn more about them on GeeksforGeeks and Programiz.

I hope this helps!

\bibitem{AI2}
OpenAl Enie (Nov 5, 2023 version, Ernie 4.0)\\
Query1: <insert the exact wording of any subsequent input into the Al tool>\\
Output: <insert the complete output from the second query>

\bibitem{AI3}
Github CoPilot (Feb 3, 2024 version)\\
Query1: <insert the exact wording you input into the Al tool>\\
Output: <insert the complete output fron the Al tool>

\bibitem{AI4}
Google Bard (Feb 2,2024 version)\\
Query: <insert the exact wording of your query>\\
Output: <irsert the complete output from the Al tool>

\bibitem{AI5}
Bing AI\\
Query: write a FFT algorithm\\
Output: Sure, here is a Python implementation of the Fast Fourier Transform (FFT) algorithm:
\begin{lstlisting}[language=python]
import numpy as np

def fft(x):
    n = len(x)
    if n == 1:
        return x
    even = fft(x[0::2])
    odd = fft(x[1::2])
    T = [np.exp(-2j * np.pi * k / n) * odd[k] for k in range(n // 2)]
    return [even[k] + T[k] for k in range(n // 2)] + [even[k] - T[k] for k in range(n // 2)]
\end{lstlisting}
The FFT is a widely used algorithm for computing the \textbf{Discrete Fourier Transform (DFT)} of a sequence. It reduces the number of computations needed from O(n$^2$) to O(n log n) 12. The basic idea of the FFT is to apply divide and conquer. We divide the coefficient vector of the polynomial into two vectors, recursively compute the DFT for each of them, and combine the results to compute the DFT of the complete polynomial 3.

I hope this helps!

\end{ReportAiUse}

\end{document}
%% 
%% This work consists of these files mcmthesis.dtx,
%%                                   figures/ and
%%                                   code/,
%% and the derived files             mcmthesis.cls,
%%                                   mcmthesis-demo.tex,
%%                                   README,
%%                                   LICENSE,
%%                                   mcmthesis.pdf and
%%                                   mcmthesis-demo.pdf.
%%
%% End of file `mcmthesis-demo.tex'.
